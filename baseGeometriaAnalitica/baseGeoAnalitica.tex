\documentclass[12pt]{article}
%\usepackage[a2paper,landscape]{geometry}
%\usepackage[a2paper,landscape]{geometry}
%\usepackage{lipsum}
\usepackage{lmodern}
\usepackage{enumerate}
\usepackage{amsmath}
\usepackage{mathtools}
\usepackage{standalone}
\DeclarePairedDelimiter\abs{\lvert}{\rvert}
\usepackage[poster]{tcolorbox}
\tcbuselibrary{listings}
\pagestyle{empty}
\usepackage{gitinfo2}
\usepackage{graphicx}
\renewcommand{\gitMark}{Branch:\,\gitBranch\,@\,\gitAbbrevHash{}; Author:\,\gitAuthorName; Date:\,\gitAuthorIsoDate~\textbullet{}}
\makeatletter
\AddToShipoutPictureBG{%
	\AtPageLowerLeft{%
		\kern2.6cm
		\raisebox{\dimexpr.5\paperheight-.8\height}
		{\rotatebox{90}{\gitMarkFormat\gitMarkPref{} \textbullet{} \gitMark}}%
	}%
}%
\makeatother
\begin{document}
	\begin{tcbposter}[
		coverage = {
			spread,
			interior style={
				%top color=yellow,bottom color=yellow!50!red
			},
		%	watermark text={\LaTeX\ Poster},
			%watermark color=yellow,	
		},
		poster = {
			showframe=false,
			%showframe,
			columns=3,
			rows=5}
		]
		% Here, we insert the poster content later
		\posterbox{name=title,column=1,below=top}{{\Large Base Analitica} \\
		}
\posterbox[fit,adjusted title=Valore assoluto,halign=left,halign=flush left,  valign=top]{name=valass ,column=1,row=1,below=title}
{\[\abs{a}=\begin{dcases}
		+a&\quad\text{se}\quad a>0\\
		0&\quad\text{se}\quad a=0\\
		-a&\quad\text{se}\quad a<0\\
	\end{dcases}\]}
\posterbox[fit,adjusted title=Valore assoluto esempi,halign=left,halign=flush left,  valign=top]{name=valasses,column=2,row=1,below=title}
{\[\abs{+5}=5\]
\[\abs{-3}=--3=+3\]}
\posterbox[fit,adjusted title=Segmento parallelo asse x,halign=left,halign=flush left,  valign=top]{name=parassex,column=1,row=2,span=1}
{\begin{center}
\begin{tikzpicture}[line cap=round,line join=round,x=0.7cm,y=0.7cm]
	\clip(-1.,-1.) rectangle (6.,4.);
	\draw (1.,2.)-- (5.,2.);
	\draw[->] (-1.,0)-- (6.0,0);
	\draw[->] (0,-1)-- (0,4);
	\draw [fill=black] (1.,2.) circle (2.5pt);
	\draw[color=black] (1.14,2.37) node {$A$};
	\draw [fill=black] (5.,2.) circle (2.5pt);
	\draw[color=black] (5.14,2.37) node {$B$};
\end{tikzpicture}
\end{center}}
\posterbox[fit,adjusted title=Segmento parallelo asse y,halign=left,halign=flush left,  valign=top]{name=parassey,column=2,row=2,span=1}
{\begin{center}
	\begin{tikzpicture}[line cap=round,line join=round,x=0.7cm,y=0.7cm]
		\clip(-1.,-1.) rectangle (6.,4.);
		\draw (2.,1.)-- (2.,3.);
		\draw[->] (-1.,0)-- (6.0,0);
		\draw[->] (0,-1)-- (0,4);
		\draw [fill=black] (2.,1.) circle (2.5pt);
		\draw[color=black] (2.0,0.8) node {$A$};
		\draw [fill=black] (2.,3.) circle (2.5pt);
		\draw[color=black] (2.,3.3) node {$B$};
	\end{tikzpicture}
\end{center}}
\posterbox[fit,adjusted title=Segmento obliquo,halign=left,halign=flush left,  valign=top]{name=segobliquo,column=3,row=2,span=1}
{\begin{center}
		\begin{tikzpicture}[line cap=round,line join=round,x=0.7cm,y=0.7cm]
		\clip(-1.,-1.) rectangle (6.,4.);
		\draw (1.,1.)-- (4.,3.);
		\draw[->] (-1.,0)-- (6.0,0);
		\draw[->] (0,-1)-- (0,4);
		\draw [fill=black] (1.,1.) circle (2.5pt);
		\draw[color=black] (1.0,0.8) node {$A$};
		\draw [fill=black] (4.,3.) circle (2.5pt);
		\draw[color=black] (4.,3.3) node {$B$};
		\end{tikzpicture}
\end{center}}
\posterbox[fit,adjusted title=Distanza AB,%halign=left,halign=flush left,
  valign=top]{name=dparassex,column=1,row=3,span=1,below=parassex}
{$\begin{aligned}
		&A(x_1;y_1)\\
		&B(x_2;y_1)\\
		d(AB)&=\abs{x_1-x_2}
		\end{aligned}$}
\posterbox[fit,adjusted title=Distanza AB,halign=left,halign=flush left,  valign=top]{name=dparassey,column=2,row=3,span=1,below=parassey}
{$\begin{aligned}
		&A(x_1;y_1)\\
		&B(x_1;y_2)\\
		d(AB)&=\abs{y_1-y_2}
	\end{aligned}$}
\posterbox[fit,adjusted title=Distanza AB,halign=left,halign=flush left,  valign=top]{name=dparassey,column=3,row=3,below=segobliquo}
{$\begin{aligned}
		A&(x_1;y_1)\\
		B&(x_2;y_2)\\
		d(AB)=&\sqrt{(x_1-x_2)^2+(y_1-y_2)^2}
	\end{aligned}$}
\posterbox[fit,adjusted title=Punto Medio segmento parallelo asse x,halign=left,halign=flush left,  valign=top]{name=Mparassex,column=1,row=4,span=1}
{\begin{center}
		\begin{tikzpicture}[line cap=round,line join=round,x=0.7cm,y=0.7cm]
			\clip(-1.,-1.) rectangle (6.,4.);
			\draw (1.,2.)-- (5.,2.);
			\draw[->] (-1.,0)-- (6.0,0);
			\draw[->] (0,-1)-- (0,4);
			\draw [fill=black] (1.,2.) circle (2.5pt);
			\draw[color=black] (1.14,2.37) node {$A$};
			\draw [fill=black] (3.,2.) circle (2.5pt);
			\draw[color=black] (3.14,2.37) node {$M$};
			\draw [fill=black] (5.,2.) circle (2.5pt);
			\draw[color=black] (5.14,2.37) node {$B$};
		\end{tikzpicture}
\end{center}}
\posterbox[fit,adjusted title=Punto Medio segmento parallelo asse y,halign=left,halign=flush left,  valign=top]{name=Mparassey,column=2,row=4,span=1}
{\begin{center}
		\begin{tikzpicture}[line cap=round,line join=round,x=0.7cm,y=0.7cm]
			\clip(-1.,-1.) rectangle (6.,4.);
			\draw (2.,1.)-- (2.,3.);
			\draw[->] (-1.,0)-- (6.0,0);
			\draw[->] (0,-1)-- (0,4);
			\draw [fill=black] (2.,1.) circle (2.5pt);
			\draw[color=black] (2.0,0.6) node {$A$};
			\draw [fill=black] (2.,2.) circle (2.5pt);
			\draw[color=black] (2.4,2.0) node {$M$};
			\draw [fill=black] (2.,3.) circle (2.5pt);
			\draw[color=black] (2.,3.3) node {$B$};
		\end{tikzpicture}
\end{center}}
\posterbox[fit,adjusted title=Segmento obliquo,halign=left,halign=flush left,  valign=top]{name=Msegobliquo,column=3,row=4,span=1}
{\begin{center}
		\begin{tikzpicture}[line cap=round,line join=round,x=0.7cm,y=0.7cm]
			\clip(-1.,-1.) rectangle (6.,4.);
			\draw (1.,1.)-- (4.,3.);
			\draw[->] (-1.,0)-- (6.0,0);
			\draw[->] (0,-1)-- (0,4);
			\draw [fill=black] (1.,1.) circle (2.5pt);
			\draw[color=black] (1.0,0.8) node {$A$};
			\draw [fill=black] (2.5,2.) circle (2.5pt);
			\draw[color=black] (2.5,1.7) node {$M$};
			\draw [fill=black] (4.,3.) circle (2.5pt);
			\draw[color=black] (4.,3.3) node {$B$};
		\end{tikzpicture}
\end{center}}
\posterbox[fit,adjusted title=Punto medio AB,%halign=left,halign=flush left,
valign=top]{name=dparassex,column=1,row=5,span=1,below=Mparassex}
{$\begin{aligned}
		&A(x_1;y_1)\\
		&B(x_2;y_1)\\
		&M(\dfrac{x_1+x_2}{2};y_1)
	\end{aligned}$}
\posterbox[fit,adjusted title=Punto medio AB,halign=left,halign=flush left,  valign=top]{name=dparassey,column=2,row=5,span=1,below=Mparassey}
{$\begin{aligned}
		&A(x_1;y_1)\\
		&B(x_1;y_2)\\
	&M(x_1;\dfrac{y_1+y_2}{2})
	\end{aligned}$}
\posterbox[fit,adjusted title=Punto medio AB,halign=left,halign=flush left,  valign=top]{name=dparassey,column=3,row=5,below=Msegobliquo}
{$\begin{aligned}
		&A(x_1;y_1)\\
		&B(x_2;y_2)\\
		&M(\dfrac{x_1+x_2}{2};\dfrac{y_1+y_2}{2})
	\end{aligned}$}
\end{tcbposter}
\end{document}