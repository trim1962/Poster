\documentclass[10pt,a4paper,final]{article}
\usepackage[utf8]{inputenc}
\usepackage[T1]{fontenc}
\usepackage[italian]{babel}
\usepackage{amsmath}
\usepackage{amsfonts}
\usepackage{amssymb}
\usepackage{graphicx}
\usepackage{microtype}
\author{claudio Duchi}
\title{Quadrato binomio}
\usepackage{smartdiagram}
\usesmartdiagramlibrary{additions}
\usepackage{gitinfo2}
\renewcommand{\gitMark}{Branch:\,\gitBranch\,@\,\gitAbbrevHash{}; Author:\,\gitAuthorName; Date:\,\gitAuthorIsoDate~\textbullet{}}
\makeatletter
\AddToShipoutPictureBG{%
	\AtPageLowerLeft{%
		\kern2.6cm
		\raisebox{\dimexpr.5\paperheight-.8\height}
		{\rotatebox{90}{\gitMarkFormat\gitMarkPref{} \textbullet{} \gitMark}}%
	}%
}%
\makeatother
\begin{document}
	\section*{Da periodico misto a frazione}
\begin{center}
	\smartdiagramset{set color list={white,white,white,white,white,white},
	}
	\smartdiagram[descriptive diagram]{
		{1,{Inizio}},
		{2,{Leggo il numero decimale periodico}},
		{3, {Tolgo la virgola}},
		{4, Scrivo il numero  usando le cifre della parte intera e dell'antiperiodo e del periodo},
		{5, {Sottraggo al numero la sua parte intera e scrivo il risultato al numeratore}},{6,{Scrivo a denominatore tanti nove quante sono le cifre del periodo}},{7,{Fine}},
	}
\end{center}
\end{document}