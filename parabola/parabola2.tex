\documentclass[12pt]{article}
%\usepackage[a2paper,landscape]{geometry}
%\usepackage[a2paper,landscape]{geometry}
%\usepackage{lipsum}
\usepackage{lmodern}
\usepackage{enumerate}
\usepackage[fleqn]{amsmath}
\usepackage{tikz}
\usetikzlibrary{shapes,arrows}
%\setlength{\mathindent}{0pt}
%\usepackage{etoolbox}
%\newcommand{\zerodisplayskips}{%
%	\setlength{\abovedisplayskip}{0pt}%
%	\setlength{\belowdisplayskip}{0pt}%
%	\setlength{\abovedisplayshortskip}{0pt}%
%	\setlength{\belowdisplayshortskip}{0pt}}
%\appto{\normalsize}{\zerodisplayskips}
%\appto{\small}{\zerodisplayskips}
%\appto{\footnotesize}{\zerodisplayskips}
\usepackage[poster]{tcolorbox}
\tcbuselibrary{listings}
\pagestyle{empty}
\usepackage{gitinfo2}
\usepackage{graphicx}
\renewcommand{\gitMark}{Branch:\,\gitBranch\,@\,\gitAbbrevHash{}; Author:\,\gitAuthorName; Date:\,\gitAuthorIsoDate~\textbullet{}}
\makeatletter
\AddToShipoutPictureBG{%
	\AtPageLowerLeft{%
		\kern2.6cm
		\raisebox{\dimexpr.5\paperheight-.8\height}
		{\rotatebox{90}{\gitMarkFormat\gitMarkPref{} \textbullet{} \gitMark}}%
	}%
}%
\makeatother
\begin{document}
	\begin{tcbposter}[
		coverage = {
			spread,
			interior style={
				%top color=yellow,bottom color=yellow!50!red
			},
		%	watermark text={\LaTeX\ Poster},
			%watermark color=yellow,	
		},
		poster = {
			showframe=false,
			%showframe,
			columns=3,
			rows=5}
		]
		% Here, we insert the poster content later
		\posterbox{name=title,column=1,below=top}{
			\bfseries\Huge Parabola\\
		}
\posterbox[fit,adjusted title=Equazione,halign=left,halign=flush left,  valign=top]{name=parabola,column=2,row=1}
{\[y=ax^2+bx+c\]}
\posterbox[fit,adjusted title=Concavità,halign=left,halign=flush left,  valign=top]{name=concavita,column=1,span=1,below=parabola}
{\[\begin{cases}
		a>0& \quad\text{Rivolta verso l'alto}\\
		a<0& \quad\text{Rivolta verso il basso}
	\end{cases}\]}
\posterbox[fit,adjusted title=Concavità verso l'alto,halign=left,halign=flush left,  valign=top]{name=cornasu,column=2,row=2,span=1,below=parabola}
{\begin{center}
		\begin{tikzpicture}[line cap=round,line join=round,x=0.7cm,y=0.7cm]
		\clip(-3.,-2.5) rectangle (3.,2.5);
		\draw[color=black,smooth,samples=100,domain=-3.0:3.0] plot(\x,{1/2*(\x)^(2)-2});
\end{tikzpicture}
\end{center}}
\posterbox[fit,adjusted title=Concavità verso il basso,halign=left,halign=flush left,  valign=top]{name=cornagiu,column=3,row=2,span=1,below=parabola}
{\begin{center}
		\begin{tikzpicture}[line cap=round,line join=round,x=0.7cm,y=0.7cm]
			\clip(-3.,-2.5) rectangle (3.,2.5);
			\draw[color=black,smooth,samples=100,domain=-3.0:3.0] plot(\x,{0-1/2*(\x)^(2.0)+2});
		\end{tikzpicture}
\end{center}}
\posterbox[fit,adjusted title=Vertice,halign=left,halign=flush left,  valign=top]{name=concavita,column=1,row=3,span=1}
{\[\begin{cases}
		x=&-\dfrac{b}{2a}\\
		y=&-\dfrac{\Delta}{4a}\\
		\Delta=&b^2-4ac
	\end{cases}\]}

\posterbox[fit,adjusted title=Vertice a positiva,halign=left,halign=flush left,  valign=top]{name=verticeuno,column=2,row=3,span=1}
{\begin{center}
	\begin{tikzpicture}[line cap=round,line join=round,x=0.75cm,y=0.75cm]
		\clip(-3.,-2.5) rectangle (3.,2.5);
		\draw[color=black,smooth,samples=100,domain=-3.0:3.0] plot(\x,{1/2*(\x)^(2.0)-2});
		\draw (0.,-2.5) -- (0.,2.5);
			\draw [fill=black] (0.,-2.) circle (2.5pt);
			\draw[color=black] (0.14,-1.67) node {$V$};
	\end{tikzpicture}
\end{center}}
\posterbox[fit,adjusted title=Vertice a negativa,halign=left,halign=flush left,  valign=top]{name=verticedue,column=3,row=3,span=1}
{\begin{center}
\begin{tikzpicture}[line cap=round,line join=round,x=0.75cm,y=0.75cm]
	\clip(-3.,-2.5) rectangle (3.,2.5);
	\draw[color=black,smooth,samples=100,domain=-3.0:3.0] plot(\x,{0-1/2*(\x)^(2.0)+2});
	\draw  (0.,-2.5) -- (0.,2.5);
	
		\draw [fill=black] (0.,2.) circle (2.0pt);
		\draw[color=black] (0.14,2.33) node {$V$};
	
\end{tikzpicture}
\end{center}}
\posterbox[fit,adjusted title=Intersezione asse x. Delta positivo,halign=left,halign=flush left,  valign=top]{name=deltapos,column=1,row=4}
{\[\begin{cases}
	x_1=&\dfrac{-b+\sqrt{b^2-4ac}}{2a}\\
	x_2=&\dfrac{-b-\sqrt{b^2-4ac}}{2a}\\
	\end{cases}\]
\[A(x_1,0)\]
\[B(x_2,0)\]
}
\posterbox[fit,adjusted title=Intersezione asse x. Delta zero,halign=left,halign=flush left,  valign=top]{name=deltanul,column=2,row=4}
{\[\begin{cases}
		x_1=&\dfrac{-b}{2a}\\
		x_2=&\dfrac{-b}{2a}\\
	\end{cases}\]
\[A(x_1,0)\]}
\posterbox[fit,adjusted title=Intersezione asse x. Delta negativo,halign=left,halign=flush left,  valign=top]{name=deltaneg,column=3,row=4}
{Nessuna soluzione}
\posterbox[fit,adjusted title=Soluzioni distinte,halign=left,halign=flush left,  valign=top]{name=soldis,column=1,row=4,span=1,below=deltapos}
{\begin{center}
		\begin{tikzpicture}[line cap=round,line join=round,x=0.75cm,y=0.75cm]
			\clip(-3.,-2.5) rectangle (3.,2.5);
			\draw[color=black,smooth,samples=100,domain=-3.0:3.0] plot(\x,{1/2*(\x)^(2)-2});
			\draw [domain=-3.:3.] plot(\x,{(-0.-0.*\x)/1.});
				\draw [fill=black] (-2.,0.) circle (2.5pt);
				\draw[color=black] (-2.08,-0.29) node {$B$};
				\draw [fill=black] (2.,0.) circle (2.5pt);
				\draw[color=black] (1.98,-0.33) node {$A$};
		\end{tikzpicture}
\end{center}}
\posterbox[fit,adjusted title=Soluzioni coincidenti,halign=left,halign=flush left,  valign=top]{name=solcon,column=2,row=4,span=1,below=deltanul}
{\begin{center}
		\begin{tikzpicture}[line cap=round,line join=round1,x=0.75cm,y=0.75cm]
			\clip(-3.,-2.5) rectangle (3.,2.5);
			\draw[color=black,smooth,samples=100,domain=-3.0:3.0] plot(\x,{1/2*(\x)^(2)-2});
			\draw [domain=-3.:3.] plot(\x,{(-2.-0.*\x)/1.});
			
				\draw [fill=black] (0.,-2.) circle (2.0pt);
				\draw[color=black] (-0.08,-2.29) node {$A$};

		\end{tikzpicture}
\end{center}}
\posterbox[fit,adjusted title=Nessuna Soluzione,halign=left,halign=flush left,  valign=top]{name=nosol,column=3,row=4,span=1,below=deltaneg}
{\begin{center}
		\begin{tikzpicture}[line cap=round,line join=round,x=0.75cm,y=0.75cm]
			\clip(-3.,-2.5) rectangle (3.,2.5);
			\draw[color=black,smooth,samples=100,domain=-3.0:3.0] plot(\x,{1/2*(\x)^(2)-2});
			\draw [domain=-3.:3.] plot(\x,{(-2.3-0.*\x)/1.});
		\end{tikzpicture}
\end{center}}

	\end{tcbposter}
\begin{tcbposter}[
	coverage = {
		spread,
		interior style={
			%top color=yellow,bottom color=yellow!50!red
		},
		%	watermark text={\LaTeX\ Poster},
		%watermark color=yellow,	
	},
	poster = {
		showframe=false,
		%showframe,
		columns=3,
		rows=5}
	]
	% Here, we insert the poster content later
%	\posterbox{name=title,column=1,below=top}{
%		\bfseries\Huge Parabola\\
%	}
	\posterbox[fit,adjusted title=Intersezione asse y,halign=left,halign=flush left,  valign=top]{name=interAssey,column=1,row=1}
	{\[\begin{cases}
			y=&ax^2+bx+c\\
			x=&0\\
		\end{cases}\]
\[C(0,c)\]	
}
	\posterbox[fit,adjusted title=Intersezione asse y,halign=left,halign=flush left,  valign=top]{name=intAsseyG,column=2,row=1,span=1}
	{\begin{center}
		\begin{tikzpicture}[line cap=round,line join=round,x=0.7cm,y=0.7cm]
			\clip(-3.,-2.5) rectangle (3.,2.5);
			\draw[color=black,smooth,samples=100,domain=-3.0:3.0] plot(\x,{1/2*(\x)^(2)-2});
			\draw [->] (2.5,-2.5) -- (2.5,2.5);
		
				\draw [fill=black] (2.5,1.125) circle (2.5pt);
				\draw[color=black] (2.8,1.11) node {$V$};
		\end{tikzpicture}
	\end{center}}
\end{tcbposter}
\tikzstyle{decision} = [diamond, draw, fill=blue!20,
text width=4.5em, text badly centered, node distance=2.5cm, inner sep=0pt]
\tikzstyle{block} = [rectangle, draw, %fill=blue!20,
text width=5em, text centered, , node distance=2.5cm,rounded corners, minimum height=3em]
\tikzstyle{line} = [draw, very thick, %color=black!50,
-latex']
\tikzstyle{cloud} = [draw, ellipse,%fill=red!20, 
node distance=2.5cm, minimum height=2em]
\section*{Disegnare una parabola}
\begin{center}
	\begin{tikzpicture}[scale=1, node distance = 1.5cm, auto]
	% Place nodes
	\node [cloud] (init) {Inizio};
	%\node [cloud, left of=init] (expert) {expert};
%	\node [cloud, right of=init] (system) {system};
	\node [block, below of=init] (passo1) {Concavità};
	\node [block, below of=passo1] (passo2) {Vertice};
	\node [block, below of=passo2] (passo3) {Intersezione asse x};
	\node [block, below of=passo3] (passo4) {Intersezione asse y};
	\node [block, below of=passo4] (passo5) {Trovare i punti};
	\node [block, below of=passo5] (passo6) {Disegnare Parabola};
	\node [cloud, below of=passo6] (fine) {Fine};
	\path [line] (init) -- (passo1);
	\path [line] (passo1) -- (passo2);
	\path [line] (passo2) -- (passo3);
		\path [line] (passo3) -- (passo4);
		\path [line] (passo4) -- (passo5);
		\path [line] (passo5) -- (passo6);
		\path [line] (passo6) -- (fine);
%	\node [block, below of=identify] (evaluate) {evaluate candidate models};
%	\node [block, left of=evaluate, node distance=3cm] (update) {update model};
%	\node [decision, below of=evaluate] (decide) {is best candidate better?};
%	\node [block, below of=decide, node distance=3cm] (stop) {stop};
%	% Draw edges
%	\path [line] (init) -- (identify);
%	\path [line] (identify) -- (evaluate);
%	\path [line] (evaluate) -- (decide);
%	\path [line] (decide) -| node [near start] {yes} (update);
%	\path [line] (update) |- (identify);
%	\path [line] (decide) -- node {no}(stop);
%	\path [line,dashed] (expert) -- (init);
%	\path [line,dashed] (system) -- (init);
%	\path [line,dashed] (system) |- (evaluate);
\end{tikzpicture}
\end{center}
\end{document}