\documentclass[10pt,a4paper,tikz,border=20pt]{standalone}
\usepackage[utf8]{inputenc}
\usepackage[T1]{fontenc}
\usepackage{amsmath}
%\usepackage{amsfonts}
\usepackage{amssymb}
\usepackage{mathtools}
\DeclarePairedDelimiter\abs{\lvert}{\rvert}
\usepackage{tikz}
\usetikzlibrary{shapes,arrows,shapes.misc,shapes.geometric,positioning}
\usepackage{gitinfo2}
\usepackage{graphicx}
\renewcommand{\gitMark}{Branch:\,\gitBranch\,@\,\gitAbbrevHash{}; Author:\,\gitAuthorName; Date:\,\gitAuthorIsoDate~\textbullet{}}
\makeatletter
\AddToShipoutPictureBG{%
	\AtPageLowerLeft{%
		\kern2.6cm
		\raisebox{\dimexpr.5\paperheight-.8\height}
		{\rotatebox{90}{\gitMarkFormat\gitMarkPref{} \textbullet{} \gitMark}}%
	}%
}%
\makeatother
\begin{document}
	\tikzset{
		decision/.style={diamond, draw, %fill=blue!20,
			text width=4.5em, text badly centered, 
			node distance=2.5cm, inner sep=0pt
		},
		block/.style={rectangle, draw, %fill=blue!20,
			text width=15em, text centered, 
			node distance=1.5cm,
			%rounded corners, 
		%	minimum height=3em
		},
		loop/.style={chamfered rectangle,chamfered rectangle 	xsep=2cm, draw, %fill=blue!20,
			text width=15em, text centered,  
			%node distance=2.5cm,% minimum height=3em
		},
	cloud/.style={draw, ellipse,%fill=red!20, 
		node distance=2.5cm, minimum height=2em
	},
	input/.style={ % requires library shapes.geometric
		draw,
		trapezium ,
		trapezium left angle=60,
		trapezium right angle=120,node distance=2.5cm,
	},
	line/.style={draw, very thick, %color=black!50,
		-latex'}	
	}
%	\begin{center}
		\begin{tikzpicture}[scale=1, 
			%node distance = 2.0cm, 
			auto]
			% Place nodes
			\node [cloud] (init) {Inizio};
			\node [input, below of=init] (passo1) {Leggo le coordinate dei punti};
			\node [block, below of=passo1] (passo2) {Calcolo lunghezze dei lati $a$, $b$ e $c$ };
			\node [block, below of=passo2] (passo3) {Calcolo il perimetro $2p$};
			\node [block, below of=passo3] (passo4) {Calcolo il semiperimetro $p$};
			\node [block, below of=passo4] (passo5) {Calcolo $p-a$};
			\node [block, below of=passo5] (passo6) {Calcolo $p-b$};
			\node [block, below of=passo6] (passo7) {Calcolo $p-c$};
			\node [block, below of=passo7] (passo8) {Calcolo\\ $p(p-a)(p-b)(p-c)$};
				\node [block, below of=passo7] (passo8) {Calcolo\\ $p(p-a)(p-b)(p-c)$};
					\node [block, below of=passo8] (passo9) {Area=\\ $\sqrt{p(p-a)(p-b)(p-c)}$};
					\node [cloud, below of=passo9] (passo10) {Fine};	

			\path [line] (init) -- (passo1);
			\path [line] (passo1) -- (passo2);
			\path [line] (passo2) -- (passo3);
			\path [line] (passo3) -- (passo4);
			\path [line] (passo4) -- (passo5);
			\path [line] (passo5) -- (passo6);
			\path [line] (passo6) -- (passo7);
			\path [line] (passo7) -- (passo8);
			\path [line] (passo8) -- (passo9);
			\path [line] (passo9) -- (passo10);
		\end{tikzpicture}
%	\end{center}
\end{document}